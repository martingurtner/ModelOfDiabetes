%\documentclass[14pt]{article}
\documentclass{article}
%\usepackage{extsizes}
\usepackage{amsfonts}
\usepackage{amsmath}
\usepackage[a4paper, inner=1.5cm, outer=2.5cm, top=2cm,
bottom=2cm, bindingoffset=1cm]{geometry}
\usepackage{graphicx}
\usepackage{subfigure}
\graphicspath{{img/}}  
\usepackage[section]{placeins}
\usepackage{afterpage}
\usepackage{titling}
\usepackage{siunitx}
\usepackage{epstopdf}
\usepackage{tikz}
\usepackage{bondgraph}
%\usepackage{cite}
\usepackage{booktabs}
\usepackage{color, colortbl}
\definecolor{Gray}{gray}{0.95}


\providecommand{\e}[1]{\ensuremath{\times 10^{#1}}}
\providecommand{\m}[1]{\ensuremath{\mathrm{#1}}}
\providecommand{\p}[1]{\ensuremath{\partial #1}}

\newcommand{\figWidth}{0.5\textwidth}
\newcommand{\subfigWidth}{0.45\textwidth}

\usepackage{todonotes}
\usetikzlibrary{shapes,arrows,decorations.markings}

\begin{document}

\title{Insulin-glucose interaction models}
\author{J\'achym Barv\'inek, Ji\v r\'i Figura, Martin Gurtner}

\maketitle


\emph{Assumptions about the system:}
\begin{enumerate}
	\item
	Insulin is being degraded by the liver.	
	\item
	A rise in the concentration of glucose in the bloodstream results in increased production of insulin by the pancreas.
	\item
	With increasing concentration of insulin, the glucose is more easily absorbed by the cells, i.e. it leaves the bloodstream more quickly.
	\item
	A rise in glucose concentration in the bloodstream results in increased glucose absorption by the liver.

\end{enumerate}


\emph{Bollie's model of glucose-insulin system, general form:}
\begin{equation*}
\label{eqbolie}
\begin{aligned}
V\frac{\m{d} H}{\m{d} t}&=-F_1(H)+F_2(G)+ x,\\
V\frac{\m{d} G}{\m{d} t}&=-F_3(H,G)-F_4(H,G) + y. 
\end{aligned}
\end{equation*}

\emph{Linearized form:}
\begin{equation*}
\label{eqLinearized}
\begin{aligned}
\frac{\m{d}h}{\m{d}t}&=-\alpha h + \beta g,\\
\frac{\m{d}g}{\m{d}t}&=-\gamma h - \delta g.\\
\end{aligned}
\end{equation*}



\begin{table}[h]
\renewcommand{\arraystretch}{0.8}  
\centering
\begin{tabular}{lll}
\toprule
\textbf{Symbol}  & \textbf{Meaning} & \textbf{Dimension}\\
\midrule
$V$ & volume & \si{\liter}\\
\rowcolor{Gray}
$x$ & rate of insulin injection & units \si{\per\hour}\\

$y$ & rate of glucose injection & \si{\gram\per\hour}\\
\rowcolor{Gray}
$H$ & insulin concentration & units \si{\per\liter}\\
\rowcolor{Gray}
$h$ & insulin concentration changes & units \si{\per\liter}\\
$G$ & glucose concentration & \si{\gram\per\liter}\\
\rowcolor{Gray}
$g$ & glucose concentration changes & \si{\gram\per\liter}\\
\rowcolor{Gray}
$F_1(H)$ & rate of insulin destruction & units \si{\per\hour}\\
$F_2(G)$ & rate of insulin production & units \si{\per\hour}\\

\rowcolor{Gray}
$F_3(H,G)$ & rate of liver accumulation of glucose & \si{\gram\per\hour}\\
$F_4(H,G)$ & rate of tissue utilization of glucose & \si{\gram\per\hour}\\
\bottomrule
\end{tabular}
\caption{Diabetes model parameters.}
\label{tabParam}
\end{table}



\emph{Bellomo's model:}
\begin{equation*}
\label{Eq:bellomo}
\begin{aligned}
 \frac{\m{d}i}{\m{d}t} &= -\underbrace{K_{\mathrm{i}}\,i}_{F_1} + \underbrace{K_{\mathrm{g}}\,(g - g_{\mathrm{d}})}_{F_2} + \underbrace{K_{\mathrm{s}}\,i_{\mathrm{r}}}_{x}, \\
 \frac{\m{d}g}{\m{d}t} &= \underbrace{K_{\mathrm{h}}\,g - K_0\,g\,i - K_{\mathrm{s}}\,K_{\mathrm{f}}}_{-F_3-F_4} + \underbrace{0}_{y}.
\end{aligned}
\end{equation*}

\emph{Overall glucose concentration in bloodstream according to Ackerman:}
\[G(t) = G_0 + C\,e^{-\sigma t}\cos ( \omega t  - \varphi) \]

\end{document}
