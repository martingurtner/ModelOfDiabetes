\documentclass{article}
\usepackage{amsfonts}
\usepackage{amsmath}
\usepackage[a4paper, inner=1.5cm, outer=2.5cm, top=2cm,
bottom=2cm, bindingoffset=1cm]{geometry}
\usepackage{graphicx}
\graphicspath{{img/}}  
\usepackage[section]{placeins}
\usepackage{afterpage}
\usepackage{titling}
\usepackage{siunitx}
\usepackage{epstopdf}
\usepackage{subcaption}
\usepackage{tikz}
\usepackage{bondgraph}
%\usepackage{cite}
\usepackage{booktabs}
\usepackage{color, colortbl}
\definecolor{Gray}{gray}{0.95}

\providecommand{\e}[1]{\ensuremath{\times 10^{#1}}}
\providecommand{\m}[1]{\ensuremath{\mathrm{#1}}}
\providecommand{\p}[1]{\ensuremath{\partial #1}}

\usepackage{listings}
\usepackage{todonotes}
\usetikzlibrary{shapes,arrows,decorations.markings}

\begin{document}
\title{Models for the detection of Diabetes}
\author{J\'achym Barv\'inek, Ji\v r\'i Figura, Martin Gurtner}

\maketitle

% ------------------------- Bolie's diabetes model  -------------------------
\section{Bolie's diabetes model}
\todo[inline]{Jachym dopise teoreticky uvod}

\begin{align}
V\frac{\m{d} H}{\m{d} t}&=-F_1(H)+F_2(G)+ x\\
V\frac{\m{d} G}{\m{d} t}&=-F_3(H,G)-F_4(H,G) + y 
\end{align}

For changes around equilibrium:
\todo[inline]{ vi. bod 1 dopsat assumptions}
\begin{align}
V\frac{\m{d}h}{\m{d}t}&=-F_1(H_0+h)+F_2(G_0+g)+ x\\
V\frac{\m{d}g}{\m{d}t}&=-F_3(H_0+h,G_0+g)-F_4(H_0+h,G_0+g)+ y
\end{align}

Linearization:
\begin{align}
\frac{\m{d}h}{\m{d}t}&=-\underbrace{%
		\frac{1}{V}\frac{\p F_1(H_0)}{\p H}}_{\alpha}h+%
%
\underbrace{%
		\frac{1}{V}\frac{\p F_2(G_0)}{\p G}}_{\beta}g+O(2)+\dots \\%		
%
\frac{\m{d}g}{\m{d}t}&=-\frac{1}{V}\frac{\p F_3(H_0, G_0)}{\p H}h- \frac{1}{V}\frac{\p F_3(H_0, G_0)}{\p G}g-\frac{1}{V}\frac{\p F_4(H_0, G_0)}{\p H}h-\frac{1}{V}\frac{\p F_4(H_0, G_0)}{\p G}g+\mathcal{O}(2)+\dots\\%
%
&=-\underbrace{%
\left(\frac{1}{V}\frac{\p F_3(H_0, G_0)}{\p H}+\frac{\p F_4(H_0, G_0)}{\p H}\right)}_{\gamma}h%
-\underbrace{%
\left(\frac{1}{V}\frac{\p F_3(H_0, G_0)}{\p G}+\frac{\p F_4(H_0, G_0)}{\p G}\right)}_{\delta} g+\mathcal{O}(2)+\dots
\end{align}

Linearized:
\begin{align*}
\frac{\m{d}h}{\m{d}t}=-\alpha h + \beta g\\
\frac{\m{d}g}{\m{d}t}=-\gamma h - \delta g\\
\end{align*}

\begin{table}[!h]
\renewcommand{\arraystretch}{1.3}  
\centering
\begin{tabular}{lll}
\toprule
\textbf{Symbol}  & \textbf{Meaning} & \textbf{Dimension}\\
\midrule
$V$ & volume & \si{\liter}\\
\rowcolor{Gray}
$x$ & rate of insulin injection & units \si{\per\hour}\\

$y$ & rate of glucose injection & \si{\gram\per\hour}\\
\rowcolor{Gray}
$H$ & insulin concentration & units \si{\per\liter}\\
$H_0$ & insulin concentration equilibrium & units \si{\per\liter}\\
\rowcolor{Gray}
$h$ & insulin concentration changes & units \si{\per\liter}\\
$G$ & glucose concentration & \si{\gram\per\liter}\\
\rowcolor{Gray}
$G_0$ & glucose concentration equilibrium & \si{\gram\per\liter}\\
$g$ & glucose concentration changes & \si{\gram\per\liter}\\
\rowcolor{Gray}
$F_1(H)$ & rate of insulin destruction & units \si{\per\hour}\\
$F_2(G)$ & rate of insulin production & units \si{\per\hour}\\

\rowcolor{Gray}
$F_3(H,G)$ & rate of liver accumulation of glucose & \si{\gram\per\hour}\\
$F_4(H,G)$ & rate of tissue utilization of glucose & \si{\gram\per\hour}\\
\bottomrule
\end{tabular}
\caption{Diabetes model parameters.}
\label{tabParam}
\end{table}

\todo[inline]{diskuze znamenek (bod 2), tabulka s popisem velicin}

\section{Linearized model solution}
\todo[inline]{bod 3, redukce na 2nd order, char. polynom, reseni; stabilita, kdy je stabilni periodicky/aperiodicky}

\begin{equation}
\ddot g+(\alpha+\delta)\dot g+(\beta \gamma+\delta \alpha)g=S(t)
\end{equation}

\begin{equation}
\lambda^2+(\alpha+\delta)\lambda+(\beta\gamma+\delta\alpha)=0
\end{equation}

\begin{equation}
\lambda_{1,2}=\frac{1}{2}\left(-(\alpha+\delta)\pm \sqrt{(\alpha-\delta)^2-4\beta\gamma}\right)
\end{equation}

Since both $(\alpha+\delta)$ and $(\beta\gamma+\delta\alpha)$ are positive, the solutions are always stable, going to zero with $t\rightarrow \infty$.

\section{Bolie's diabetes test}
predpoklada kriticky tlumene reseni
\todo[inline]{bod 4, povidani o testu}
\todo[inline]{ploty reseni g,h, diskuze viz. 4}
\todo[inline]{nabizi se vlozit reseny priklad, napr. 4 na str. 108}

\section{Ackermann's diabetes test}
predpoklada kmitave reseni
\todo[inline]{opsat par rovnic ze strany 105/106}
\todo[inline]{vyresit ten stejny priklad jako vyse}


%\begin{figure}[h]
%        \centering
%        \begin{subfigure}[b]{0.5\textwidth}
%        		\centering
%                \includegraphics[width=\textwidth]{11a}
%                \caption{}
%                \label{fig11a}
%        \end{subfigure}%
%        %
%        \begin{subfigure}[b]{0.5\textwidth}
%        		\centering
%                \includegraphics[width=\textwidth]{11b}
%                \caption{}
%                \label{fig11b}
%        \end{subfigure}        
%        
%        \caption{Angular velocity data and its integration.}
%        \label{figTask10}
%\end{figure}



\end{document}
